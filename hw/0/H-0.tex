\documentclass{article}
\usepackage{../fasy-hw}
\usepackage{ wasysym }

%% UPDATE these variables:
\renewcommand{\hwnum}{0}
\title{Discrete Structures, Homework 0}
\author{Jonathan Neuman}
\collab{n/a}
\date{due: 15 January 2021}

\begin{document}

\maketitle

This homework assignment should be
submitted as a single PDF file both to D2L and to Gradescope.

General homework expectations:
\begin{itemize}
    \item Homework should be typeset using LaTex.
    \item Answers should be in complete sentences and proofread.
    \item You will not plagiarize.
    \item List collaborators at the start of each question using the
        \texttt{collab} command.
\end{itemize}

% ============================================
% ============================================
\nextprob{Getting to Know You}
\collab{n/a}
% ============================================
% ============================================

Answer the following questions:
\begin{enumerate}
    \item What is your elevator pitch?  Describe yourself in 1-2
        sentences.
        \paragraph{Answer} My name is Jonathan Neuman. I am a 23 year old computer science major that loves all things computers, especially programming and gaming.

    \item What was your favorite college class so far, and why?
        \paragraph{Answer} My favorite class so far was Web Design. I enjoyed the blend of both programming and design principles and gained valuable skills in HTML and CSS.

    \item What was your least favorite college class so far, and why?
        \paragraph{Answer} My least favorite class so far was Multimedia Dev Methods. The recorded lectures online were terrible quality and I ended up using outside resources in order to learn the material as opposed to the professor's videos.

    \item Why are you interested in taking this course? (If your answer is
        `because I am required to by my major/minor', perhaps answer the
        alternative question: Why are you in your major?)
        \paragraph{Answer} To be honest I am not sure exaclty what this course entails. I am however interested in taking it because I will learn skills that make me a better computer scientist.

    \item What is your biggest academic or research goal for this semester (can
        be related to this course or not)?
        \paragraph{Answer} I want to get A's in all of my classes this semester.

    \item What do you want to do after you graduate?
        \paragraph{Answer} I want to pursue a career in either software development or web development. I am also potentially interested in graduate school.

    \item What was the most challenging aspect of blended or online courses?
        \paragraph{Answer} The most challenging aspect of online courses for me was staying focused and motivated to do my work and watch the videos. All of the classes I have taken online so far have been asynchronous and not having a set time every day for classes was a struggle for me.

    \item What do you like about blended or online courses?
        \paragraph{Answer} I enjoy not having to drive to campus and being able to continue my education while staying safe from COVID-19.

\end{enumerate}

% ============================================
% ============================================
\nextprob{Administrative Tasks}
\collab{n/a}
% ============================================
% ============================================

Please do the following:
\begin{enumerate}
    \item Write this homework in LaTex. This will not be strictly enforced for
        this homework, but it is strongly encouraged.  Future homeworks will not
        be graded if they are not typeset in LaTex.
    \item Update your photo on D2L to be a recognizable headshot of you.
    \item Sign up for the class discussion board.
\end{enumerate}

\paragraph{Answer}


I have completed all of these tasks.

% ============================================
% ============================================
\nextprob{Plagiarism}
\collab{n/a}
% ============================================
% ============================================


In this class, please properly cite all resources that you use.  To refresh your
memory on what plagiarism is, please complete the plagiarism tutorial found
here: \url{http://www.lib.usm.edu/plagiarism_tutorial}.  If you have observed
plagiarism or cheating in a classroom (either as an instructor or as a student),
explain the situation and how it made you feel.  If you have not experienced
plagiarism or cheating or if you would prefer not to reflect on a personal
experience, find a news article about plagiarism or cheating and explain how you
would feel if you were one of the people involved.

\paragraph{Answer}

When I was in highschool, the student in front of me was cheating on the final exam in Geometry. They were  looking up answers to problems on their phone and were hiding it under their desk. I  felt very pissed off that they were getting away with it because I had spent hours studying and memorizing all the the trigonometric identities and they just got to look them up without doing any of the work. Luckily, someone else reported them after the exam was over and they were severely punished for cheating. 

% ============================================
% ============================================
\nextprob{Exams}
\collab{n/a}
% ============================================
% ============================================

I am exploring various options for exams for this semester: take-home,
in-person, synchronous online.  If you have any comments about what worked or
did not work in previous semesters with respect to classes in blended and online
settings, please share that here.

\paragraph{Answer}


All of the classes I have taken online so far have had take home exams. They have worked fine and I had no issues with them. I cannot do in- person exams this semester due to my living arangements so I would prefer they happen online.




% ============================================
% ============================================
\nextprob{Terminology}
\collab{\todo{}}
% ============================================
% ============================================

Sometimes concepts are taught more than once throughout the curriculum.  Each
time you encounter a concept, your understanding of it is deepened.
For each of the terms or statements below, describe in your own words what they
mean.  This will not be graded for correctness, just whether you have done it or
not.  Answering these to the best of your ability will help the instructor and
TA understand the base knowledge of the students in this class.
I encourage you to meet with a partner or two to refresh yourself on what these
terms mean (if you do, be sure to update the \texttt{collab} command
above!).  However, please keep the web searches to a minimum for this one!  It
is acceptable to answer `I have not heard of this term' or `I have heard of
this, but do not remember what it means.'
\begin{enumerate}
    \item $f(n)$ is $O(n^2)$.
        \paragraph{Answer}
    f of n is big O for$( n^2)$
    \item $f(n)$ is $O(g(n))$.
        \paragraph{Answer}
     f of n is big O for g of n
    \item $f(n)$ is $\Omega(n^3)$.
        \paragraph{Answer}
        I know this has something to do with big O notation but I am not familiar with it.
    \item $f(n)$ is $\Theta(n\log n)$.
        \paragraph{Answer}
          I know this has something to do with big O notation but I am not familiar with it.
    \item Binomial Coefficients
      \paragraph {Answer}
        I have not heard of this term.
    \item Four Color Theorem
      \paragraph {Answer}
         I have not heard of this term.
    \item Graph
        \paragraph{Answer}
        A visual representation of data.
    \item Modus Ponens
        \paragraph{Answer}
         I have not heard of this term.
    \item Proof by Counter-example
        \paragraph{Answer}
         I have not heard of this term.
    \item Proof by Example
        \paragraph{Answer}
         I have not heard of this term.
    \item Proof by Induction
        \paragraph{Answer}
         I have not heard of this term.
    \item Recurrence Relation
        \paragraph{Answer}
         I have not heard of this term.
    \item Recursive Algorithm
        \paragraph{Answer}
       An algorithm that calls itself.
    \item Searching Algorithms
        \paragraph{Answer}
        An algorithm that searches a data structure for information.
    \item Sorting Algorithms
        \paragraph{Answer}
        An algorithm that sorts data in a data structure.
    \item Tree
        \paragraph{Answer}
        An abstract data type that has a tree structure.
\end{enumerate}

% ============================================
% ============================================
\nextprob{Real Numbers}
\collab{\todo{}}
% ============================================
% ============================================

Review the Properties of Real Numbers in Appendix A.  If any are unfamiliar or
confusing, please post a question in the group discussion board.  In the
write-up, write the following: `I have reviewed all properties of real numbers
in Appendix A.`

\paragraph{Answer}

I have reviewed this section.


% ============================================
% ============================================
\nextprob{Georg Cantor}
\collab{\todo{}}
% ============================================
% ============================================

Write a short (1-2 paragraph) biography of Georg Cantor.
\textbf{In your own words}, describe who they are and why they are important in
the history of computer science.  If you use external resources, please provide
proper citations.

\paragraph{Answer}

% ============================================

Georg Cantor was a German mathematician known for his advances in the field of mathematics. He was born in 1845 in Saint Peterburg, Russia, and died in 1918 in Halle, Germany. He is well known for his work on developing set theory. Set theory is 

% ============================================

\end{document}

